\documentclass{article}
\usepackage[margin=1in]{geometry}
\usepackage{amsmath}

\begin{document}

\begin{equation}
	x_{rx}(t) = e^{j\theta}x_i(t) = e^{j\theta}\left[(1+\varepsilon)e^{j(\phi-\theta)}\cos(2{\pi}f_ot)+je^{-j\theta}\sin(2{\pi}f_ot)\right]
\end{equation}

\begin{equation}
	x_i(t) = (1+\varepsilon)[\cos(\phi-\theta)+j\sin(\phi-\theta)]\cos(2{\pi}f_ot)+j[\cos\theta-j\sin\theta]\sin(2{\pi}f_ot)
\end{equation}

\begin{equation}
\begin{split}
	x_i(t) &= \left[(1+\varepsilon)\cos(\phi-\theta)\cos(2{\pi}f_ot)+\sin\theta\sin(2{\pi}f_ot)\right]\\
	&+j\left[(1+\varepsilon)\sin(\phi-\theta)\cos(2{\pi}f_ot)+\cos\theta\sin(2{\pi}f_ot)\right]
\end{split}
\end{equation}

\begin{equation}
\begin{split}
	x_i(t) &= \left[(1+\varepsilon)\cos(\phi-\theta)\cos(2{\pi}f_ot)+\sin\theta\cos(2{\pi}f_ot-\pi/2)\right]\\
	&+j\left[(1+\varepsilon)\sin(\phi-\theta)\sin(2{\pi}f_ot+\pi/2)+\cos\theta\sin(2{\pi}f_ot)\right]
\end{split}
\end{equation}

\begin{equation}
	x_i(t) = \alpha_1\cos(2{\pi}f_ot+\beta_1)+j\alpha_2\sin(2{\pi}f_ot+\beta_2)
\end{equation}

\begin{equation}
	\alpha_1 = \sqrt{(1+\varepsilon)^2\cos^2(\phi-\theta)+\sin^2\phi}
\end{equation}

\begin{equation}
	\alpha_2 = \sqrt{(1+\varepsilon)^2\sin^2(\phi-\theta)+\cos^2\phi}
\end{equation}

\begin{equation}
	\beta_1 = \text{atan2}(-\sin\theta,(1+\varepsilon)\cos(\phi-\theta))
\end{equation}

\begin{equation}
	\beta_2 = \text{atan2}((1+\varepsilon)\sin(\phi-\theta),\cos\theta)
\end{equation}

Want $\beta_1 = \beta_2$.

\begin{equation}
	\begin{cases}
		-k\sin\theta=(1+\varepsilon)\sin(\phi-\theta)\\
		k(1+\varepsilon)cos(\phi-\theta)=\cos\theta
	\end{cases}
\end{equation}

\begin{equation}
	k = \frac{\cos\theta}{(1+\varepsilon)\cos(\phi-\theta)}
\end{equation}

\begin{equation}
	-\frac{\sin\theta\cos\theta}{(1+\varepsilon)\cos(\phi-\theta)}=(1+\varepsilon)\sin(\phi-\theta)
\end{equation}

\begin{equation}
	-\sin(2\theta)=(1+\varepsilon)^2\sin(2(\phi-\theta))
\end{equation}

\begin{equation}
	-\sin(2\theta)=(1+\varepsilon)^2\left[\sin(2\phi)\cos(2\theta) - \sin(2\theta)\cos(2\phi)\right]
\end{equation}

\begin{equation}
	\left[(1+\varepsilon)^2\cos(2\phi)-1\right]\sin(2\theta)=(1+\varepsilon)^2\sin(2\phi)\cos(2\theta)
\end{equation}

\begin{equation}
	\tan(2\theta)=\frac{(1+\varepsilon)^2\sin(2\phi)}{(1+\varepsilon)^2\cos(2\phi)-1}
\end{equation}

\begin{equation}
	2\theta = \text{atan}\left[\frac{(1+\varepsilon)^2\sin(2\phi)}{(1+\varepsilon)^2\cos(2\phi)-1}\right]+n\pi
\end{equation}

\begin{equation}
	\theta = \frac{1}{2}\text{atan}\left[\frac{(1+\varepsilon)^2\sin(2\phi)}{(1+\varepsilon)^2\cos(2\phi)-1}\right]+\frac{n\pi}{2}
\end{equation}

For phase imbalance-only:

\begin{equation}
	\theta = \frac{1}{2}\text{atan}\left[\frac{\sin(2\phi)}{\cos(2\phi)-1}\right]+\frac{n\pi}{2}
\end{equation}

\begin{equation}
	\theta = \frac{1}{2}\text{atan}\left[\frac{2\sin\phi\cos\phi}{-2\sin^2\phi}\right]+\frac{n\pi}{2}
\end{equation}

\begin{equation}
	\theta = \frac{1}{2}\text{atan}(-\text{cot}\phi)+\frac{n\pi}{2}
\end{equation}

\begin{equation}
	\theta = \frac{1}{2}\text{atan}[\text{cot}(-\phi)]+\frac{n\pi}{2}
\end{equation}

\begin{equation}
	\theta = \frac{1}{2}\text{atan}\left[\text{tan}\left(\phi+\frac{\pi}{2}\right)\right]+\frac{n\pi}{2}
\end{equation}

\begin{equation}
	\theta = \frac{\phi}{2}+\frac{\pi}{4}+\frac{n\pi}{2}
\end{equation}

\end{document}