\documentclass{article}
\usepackage[margin=1in]{geometry}
\usepackage{amsmath}

\begin{document}

To characterize the effects of I/Q imbalance on the resulting constellations, we can write our received signal in the following form:

\begin{equation}
	x_{rx}(t) = e^{j\theta}x_i(t) = e^{j\theta}\left[(1+\varepsilon)e^{j(\phi-\theta)}\cos(2{\pi}f_ot)+je^{-j\theta}\sin(2{\pi}f_ot)\right]
\end{equation}

where $\theta$ represents a rotation in the complex plane. Next, we can solve $x_i(t)$ (the signal prior to the rotation). Doing, so we get the following result:

\begin{equation}
	x_i(t) = (1+\varepsilon)[\cos(\phi-\theta)+j\sin(\phi-\theta)]\cos(2{\pi}f_ot)+j[\cos\theta-j\sin\theta]\sin(2{\pi}f_ot)
\end{equation}

\begin{equation}
\begin{split}
	x_i(t) &= \left[(1+\varepsilon)\cos(\phi-\theta)\cos(2{\pi}f_ot)+\sin\theta\sin(2{\pi}f_ot)\right]\\
	&+j\left[(1+\varepsilon)\sin(\phi-\theta)\cos(2{\pi}f_ot)+\cos\theta\sin(2{\pi}f_ot)\right]
\end{split}
\end{equation}

\begin{equation}
\begin{split}
	x_i(t) &= \left[(1+\varepsilon)\cos(\phi-\theta)\cos(2{\pi}f_ot)+\sin\theta\cos(2{\pi}f_ot-\pi/2)\right]\\
	&+j\left[(1+\varepsilon)\sin(\phi-\theta)\sin(2{\pi}f_ot+\pi/2)+\cos\theta\sin(2{\pi}f_ot)\right]
\end{split}
\end{equation}

Using phasor math, we can put $x_i(t)$ in the following form:

\begin{equation}
	x_i(t) = \alpha_1\cos(2{\pi}f_ot+\beta_1)+j\alpha_2\sin(2{\pi}f_ot+\beta_2)
\end{equation}

where $\alpha_1$, $\alpha_2$, $\beta_1$, and $\beta_2$ are given as follows:

\begin{equation}
	\alpha_1 = \sqrt{(1+\varepsilon)^2\cos^2(\phi-\theta)+\sin^2\theta}
\end{equation}

\begin{equation}
	\alpha_2 = \sqrt{(1+\varepsilon)^2\sin^2(\phi-\theta)+\cos^2\theta}
\end{equation}

\begin{equation}
	\beta_1 = \text{atan}\left[\frac{-\sin\theta}{(1+\varepsilon)\cos(\phi-\theta)}\right]
\end{equation}

\begin{equation}
	\beta_2 = \text{atan}\left[\frac{(1+\varepsilon)\sin(\phi-\theta)}{\cos\theta}\right]
\end{equation}

We are interested in values of $\theta$ where $\beta_1$ = $\beta_2$. This gives us an ellipse whose major and minor axis correspond with the real and imaginary axis. Solving for $\theta$, we get the following result:

\begin{equation}
	\frac{-\sin\theta}{(1+\varepsilon)\cos(\phi-\theta)} = \frac{(1+\varepsilon)\sin(\phi-\theta)}{\cos\theta}
\end{equation}

\begin{equation}
	-\sin\theta\cos\theta = (1+\varepsilon)^2\sin(\phi-\theta)\cos(\phi-\theta)
\end{equation}

\begin{equation}
	-\sin(2\theta) = (1+\varepsilon)^2\sin(2(\phi-\theta))
\end{equation}

\begin{equation}
	-\sin(2\theta)=(1+\varepsilon)^2\left[\sin(2\phi)\cos(2\theta) - \sin(2\theta)\cos(2\phi)\right]
\end{equation}

\begin{equation}
	\left[(1+\varepsilon)^2\cos(2\phi)-1\right]\sin(2\theta)=(1+\varepsilon)^2\sin(2\phi)\cos(2\theta)
\end{equation}

\begin{equation}
	\tan(2\theta)=\frac{(1+\varepsilon)^2\sin(2\phi)}{(1+\varepsilon)^2\cos(2\phi)-1}
\end{equation}

\begin{equation}
	2\theta = \text{atan}\left[\frac{(1+\varepsilon)^2\sin(2\phi)}{(1+\varepsilon)^2\cos(2\phi)-1}\right]
\end{equation}

\begin{equation}
	\theta = \frac{1}{2}\text{atan}\left[\frac{(1+\varepsilon)^2\sin(2\phi)}{(1+\varepsilon)^2\cos(2\phi)-1}\right]
\end{equation}

When we consider phase-imbalance only, $\theta$ can be expressed in the following form

\begin{equation}
	\theta = \frac{1}{2}\text{atan}\left[\frac{\sin(2\phi)}{\cos(2\phi)-1}\right] = \frac{1}{2}\text{atan}\left[\frac{2\sin\phi\cos\phi}{-2\sin^2\phi}\right] = \frac{1}{2}\text{atan}(-\text{cot}\phi)
\end{equation}

\begin{equation}
	\theta = \frac{1}{2}\text{atan}\left[\text{tan}\left(\phi+\frac{\pi}{2}\right)\right] = \frac{\phi}{2}+\frac{\pi}{4}
\end{equation}

Using the value we found for $\theta$, we can solve for the remaining unknowns. Doing so, we find:

\begin{align}
	\alpha_1 = \sqrt{\cos^2(\phi-\theta)+\sin^2\theta} = \sqrt{1+\sin\phi}
\end{align}

\begin{align}
	\alpha_2 = \sqrt{\sin^2(\phi-\theta)+cos^2(\theta)} = \sqrt{1-\sin\phi}
\end{align}

\begin{align}
	\beta_1 = \text{atan}\left[\frac{-\sin\theta}{\cos(\phi-\theta)}\right] = -\frac{\pi}{4}
\end{align}

\begin{align}
	\beta_2 = \text{atan}\left[\frac{\sin(\phi-\theta)}{\cos\phi}\right] = -\frac{\pi}{4}
\end{align}

Using the above equations, we can see what happens when we have I/Q phase imbalance. First, we rotate our constellation points around the unit circle by $-45^{\circ}$. Then, we stretch the unit circle along the real axis and compress it along the imaginary axis, creating an ellipse. Finally, we rotate this ellipse around the unit circle. This results in a titled ellipse as shown in Figure \ref{}.

We can perform similar analysis for I/Q amplitude imbalance. With amplitude imbalance we find that $\theta = 0$. Using the value we found for $\theta$, we can solve for the remaining unknowns. Doing so, we find:

\begin{equation}
	\alpha_1 = \sqrt{(1+\varepsilon)^2\cos^2(0)+\sin^2(0)} = 1 + \varepsilon
\end{equation}

\begin{equation}
	\alpha_2 = \sqrt{(1+\varepsilon)^2\sin^2(0)+\cos^2(0)} = 1
\end{equation}

\begin{equation}
	\beta_1 = \beta_2 = 0
\end{equation}

In other words, amplitude imbalance, only stretches the unit circle along the real axis, resulting in the ellipse we see in Figure \ref{}.

\end{document}