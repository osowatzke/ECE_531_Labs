\documentclass{article}
\usepackage[margin=1in]{geometry}
\usepackage{graphicx}
\usepackage{xcolor}
\usepackage{float}
\usepackage{amsmath}
\usepackage{cite}
\usepackage{hyperref}
\usepackage{indentfirst}
\graphicspath{{..} {./images}}

\definecolor{navy-blue}{rgb}{0.22,0.38,0.71}

\renewcommand{\contentsname}{\vspace*{-2\baselineskip}}

\hypersetup{
	colorlinks,
	linkcolor=black,
	urlcolor=blue,
	citecolor=black
}
  		
\begin{document}
\begin{titlepage}
	\centering
	{\huge Lab 6 - Digital Modulation: Carrier Synchronization}\\[0.25 in]
	\includegraphics[width=0.6\textwidth]{ua_logo.png}\\[0.25 in]
	{\large \textbf{ECE 531 - Software Defined Radio\\[0.25 in]
	April 11, 2025\\[0.25 in]}}
	{\large Owen Sowatzke, osowatzke@arizona.edu\\[0.05 in]
	Department of Electrical \& Computer Engineering\\[0.05 in]
	University of Arizona, Tucson, AZ 85721\\[0.5 in]}
	\hypersetup{linkcolor=navy-blue}
	\noindent\hrulefill
	\tableofcontents
	\noindent\hrulefill
\end{titlepage}

% \setlength{\parindent}{0pt}

\section{Introduction}
%Introduction to the laboratory experiment, including a brief description of the objectives and goals.

\section{Procedure}
% Detailed explanation of the laboratory experiment, including the design, implementation, and testing of the system.

\section{Results}
% Results and discussion of the laboratory experiment, including captured outputs, observations, and responses to laboratory questions.

\subsection{System and Error Models}

In discrete time, when we multiply a signal by a complex exponential, it circular shifts the frequency response. If we have enough excess bandwidth, we can ensure that the significant portions of our frequency response do not wrap and in turn correctly model an analog frequency shift. If we change the `filterUpsample` in \texttt{lab6part1.m} to 1, we do have any excess bandwidth. The spectrum after a frequency shift of $0.1F_s$ is shown in Figure \ref{fig::psd_upsample_1}.

\begin{figure}[H]
	\centerline{\fbox{\includegraphics[width=0.5\textwidth]{psd_upsample_1.png}}}
	\caption{Frequency Response with filterUpsample` Set to 1 and $\protect{0.1F_s}$ Frequency Shift}
	\label{fig::psd_upsample_1}
\end{figure}

\noindent Examining the frequency response, we see that it does not correctly model a $0.1F_s$ analog frequency shift, which would have resulted in part of the spectrum being zero. Note that our updated frequency response is also not related to the original frequency response by a shift. This is because we are looking at the DFT, which is sampling the DTFT. With a frequency shift of $0.1F_s$, our updated sampling positions do not overlap with the original sampling positions, which results in a slightly different response. If we instead chose a frequency shift of $F_s/16$, our sampling positions align and our DFT's are related by a simple shift. These results are captured in Figure \ref{fig::psd_upsample_1_mod_shift}.

\begin{figure}[H]
	\centerline{\fbox{\includegraphics[width=0.5\textwidth]{psd_upsample_1_mod_shift.png}}}
	\caption{Frequency Response with filterUpsample` Set to 1 and $\protect{F_s/16}$ Frequency Shift}
	\label{fig::psd_upsample_1_mod_shift}
\end{figure}

\section{Conclusion}
% Conclusions to the overall lab that discuss meaningful lessons learned and other takeaways from the assignment. (Important)

\end{document}