\documentclass{article}
\usepackage[margin=1in]{geometry}
\usepackage{graphicx}
\usepackage{xcolor}
\usepackage{float}
\usepackage{amsmath}
\usepackage{cite}
\usepackage{hyperref}
\usepackage{indentfirst}
\graphicspath{{..} {./images}}

\definecolor{navy-blue}{rgb}{0.22,0.38,0.71}

\renewcommand{\contentsname}{\vspace*{-2\baselineskip}}

\hypersetup{
	colorlinks,
	linkcolor=black,
	urlcolor=blue,
	citecolor=black
}
  		
\begin{document}
\begin{titlepage}
	\centering
	{\huge Lab 5 - Digital Modulation: Symbol Synchronization}\\[0.25 in]
	\includegraphics[width=0.6\textwidth]{ua_logo.png}\\[0.25 in]
	{\large \textbf{ECE 531 - Software Defined Radio\\[0.25 in]
	April 4, 2025\\[0.25 in]}}
	{\large Owen Sowatzke, osowatzke@arizona.edu\\[0.05 in]
	Department of Electrical \& Computer Engineering\\[0.05 in]
	University of Arizona, Tucson, AZ 85721\\[0.5 in]}
	\hypersetup{linkcolor=navy-blue}
	\noindent\hrulefill
	\tableofcontents
	\noindent\hrulefill
\end{titlepage}

% \setlength{\parindent}{0pt}

\section{Introduction}
%Introduction to the laboratory experiment, including a brief description of the objectives and goals.

\section{Procedure}
% Detailed explanation of the laboratory experiment, including the design, implementation, and testing of the system.

\section{Results}
% Results and discussion of the laboratory experiment, including captured outputs, observations, and responses to laboratory questions.

\subsection{Pulse Shaping and Matched Filtering}

Type I Nyquist filters are defined by a rectangular pulse in the frequency domain and a sinc pulse in the time domain. The sinc pulse has zero crossings at $nT$, which produces zero ISI. Because Type I Nyquist filters are a rectangular pulse in the frequency-domain, they have the minimum bandwidth and maximize spectral efficiency. However, there are a couple problems with type I Nyquist filters. First, they are infinite in time. Second, accurate sampling is required. Small timing errors can result in large ISI because the impulse response of a sinc filter decays with $1/t$, which does not converge.

Type II Nyquist filters also have zero crossing at $nT$. However, they have a bandwidth larger than the minimum bandwidth, which allows them to achieve lower ISI sensitivity than type I Nyquist filters. The raised cosine filter is an example of a type II Nyquist filter. Its impulse responses decays with $1/t^3$, which quickly converges in the presence of timing errors.

\section{Conclusion}
% Conclusions to the overall lab that discuss meaningful lessons learned and other takeaways from the assignment. (Important)
	
\end{document}