\documentclass{article}
\usepackage[margin=1in]{geometry}
\usepackage{graphicx}
\usepackage{xcolor}
\usepackage{float}
\usepackage{amsmath}
\usepackage{cite}
\usepackage{hyperref}
\graphicspath{{..} {./images}}

\definecolor{navy-blue}{rgb}{0.22,0.38,0.71}

\renewcommand{\contentsname}{\vspace*{-2\baselineskip}}

\hypersetup{
	colorlinks,
	linkcolor=black,
	urlcolor=blue,
	citecolor=black
}
  		
\begin{document}
\begin{titlepage}
	\centering
	{\huge Lab 3 - Exploring the RF Spectrum}\\[0.25 in]
	\includegraphics[width=0.6\textwidth]{ua_logo.png}\\[0.25 in]
	{\large \textbf{ECE 531 - Software Defined Radio\\[0.25 in]
	March 5, 2025\\[0.25 in]}}
	{\large Owen Sowatzke, osowatzke@arizona.edu\\[0.05 in]
	Department of Electrical \& Computer Engineering\\[0.05 in]
	University of Arizona, Tucson, AZ 85721\\[0.5 in]}
	\hypersetup{linkcolor=navy-blue}
	\noindent\hrulefill
	\tableofcontents
	\noindent\hrulefill
\end{titlepage}

\setlength{\parindent}{0pt}

\section{Introduction}
%Introduction to the laboratory experiment, including a brief description of the objectives and goals.

\section{Procedure}
% Detailed explanation of the laboratory experiment, including the design, implementation, and testing of the system.

\subsection{Identifying Signals in the RF Spectrum}

In this section, we use Fosphor to create spectrum waterfall plots of data received with the PlutoSDR. Using \url{sigidwiki.com} as a reference, we identify at least 4 different modulation techniques in received over-the-air transmissions. For each of the received signals, we describe the time and frequency characteristics that reveal the modulation technique.
 
\section{Results}

% Results and discussion of the laboratory experiment, including captured outputs, observations, and responses to laboratory questions.

\subsection{Identifying Signals in the RF Spectrum}

For our first modulation technique, we looked at the FM radio band. We specifically captured data with a center frequency of 96.1 MHz and a bandwidth of 1 MHz. Our resulting collect is shown in Figure \ref{fig::hd_radio_spectrum}.

\begin{figure}[H]
	\centerline{\fbox{\includegraphics[width=0.7\textwidth]{hd_radio_spectrum_cropped.png}}}
	\caption{HD Radio Spectrum}
	\label{fig::hd_radio_spectrum}
\end{figure}

Note that FM radio channel we collected includes HD radio sidebands. The HD radio sidebands are digital copies of the data transmitted using OFDM modulation. The analog radio spectrum is encoded with FM modulation and rapidly oscillates in frequency with maintaining a constant amplitude. For comparison, the OFDM digital sidebands occupy fixed bandwidth and peak at approximately constant values. This is expected because OFDM encodes data into each of its subcarriers using QAM, a fixed amplitude modulation scheme. 

For our next collect, we looked at the spectrum for TV modulated with 8VSB. We specifically captured data at 485 MHz with a bandwidth of 10 MHz. Our collect is shown in Figure \ref{fig::8vsb_spectrum}.

\begin{figure}[H]
	\centerline{\fbox{\includegraphics[width=0.7\textwidth]{8vsb_spectrum_cropped.png}}}
	\caption{Channel 16 TV Spectrum}
	\label{fig::8vsb_spectrum}
\end{figure}

8VSB uses 8 discrete levels of amplitude modulation applied digitally. It also employs a steep filter with steep cutoff to remove the sidelobes associated with the zero-order hold in the transition from discrete time to continuous time. This filter leads to a significant spectral savings.
 
\section{Conclusion}
% Conclusions to the overall lab that discuss meaningful lessons learned and other takeaways from the assignment. (Important)

%\nocite{analog_devices_libiio_error}
%\bibliographystyle{IEEEtran}
%\bibliography{sources}{}
%\bibliographystyle{ieeetr}
	
\end{document}