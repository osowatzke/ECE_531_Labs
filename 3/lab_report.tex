\documentclass{article}
\usepackage[margin=1in]{geometry}
\usepackage{graphicx}
\usepackage{xcolor}
\usepackage{float}
\usepackage{amsmath}
\usepackage{cite}
\usepackage{hyperref}
\graphicspath{{..} {./images}}

\definecolor{navy-blue}{rgb}{0.22,0.38,0.71}

\renewcommand{\contentsname}{\vspace*{-2\baselineskip}}

\hypersetup{
	colorlinks,
	linkcolor=black,
	urlcolor=blue,
	citecolor=black
}
  		
\begin{document}
\begin{titlepage}
	\centering
	{\huge Lab 3 - Exploring the RF Spectrum}\\[0.25 in]
	\includegraphics[width=0.6\textwidth]{ua_logo.png}\\[0.25 in]
	{\large \textbf{ECE 531 - Software Defined Radio\\[0.25 in]
	March 5, 2025\\[0.25 in]}}
	{\large Owen Sowatzke, osowatzke@arizona.edu\\[0.05 in]
	Department of Electrical \& Computer Engineering\\[0.05 in]
	University of Arizona, Tucson, AZ 85721\\[0.5 in]}
	\hypersetup{linkcolor=navy-blue}
	\noindent\hrulefill
	\tableofcontents
	\noindent\hrulefill
\end{titlepage}

\setlength{\parindent}{0pt}

\section{Introduction}
%Introduction to the laboratory experiment, including a brief description of the objectives and goals.

In this lab, we use the PlutoSDR to explore the RF spectrum. In the first part of the experiment, we use Fosphor to analyze the spectrum of 4 signals, each of which use different modulation schemes. Then, we focus specifically on 802.11 Wi-Fi, which we observe in the 2.4 GHz and 5 GHz bands. Next, we stitch together multiple collects with different center frequencies to form a picture of the full PlutoSDR spectrum, from 60 MHz to 6 GHz. Finally, we use the PlutoSDR to capture the spectrum of a personal RF device. For this part of the experiment, we specifically analyze a 2022 Honda Civic key fob. Using the spectrum, we determine the modulation scheme and propose a way to demodulate it. Finally, we determine whether the sequence of bits changes between subsequent key presses.

\section{Procedure}
% Detailed explanation of the laboratory experiment, including the design, implementation, and testing of the system.

In this section, we outline the procedures for each of the experiments that we performed. In our first experiment, we capture the spectrum of RF signals that use 4 different modulation schemes. Next, we look at the spectrum of 802.11 Wi-Fi signals collected in the 2.4 GHz and 5 GHz bands. Then, we capture the full PlutoSDR spectrum from 60 MHz to 6 GHz. Finally, we analyze the spectrum of a 2022 Honda Civic key fob and determine a method for demodulating its transmission. We also determine whether the transmitted bits change between subsequent key presses.

\subsection{Identifying Signals in the RF Spectrum}

In this experiment, we use Fosphor to create spectrum waterfall plots of data received with the PlutoSDR. Using \url{sigidwiki.com} as a reference, we identify at least 4 different modulation techniques in received over-the-air transmissions. For each of the received signals, we describe the time and frequency characteristics that reveal the modulation technique.

\subsection{IEEE 802.11 Wireless Local Area Network}

In this experiment, we examine the spectrum and waterfall plots of wireless local area networks (WLANs). We specifically capture the spectrum of the 802.11 Wi-Fi channels in the 2.4 GHz and 5 GHz band. In the 2.4 GHz band, we concentrate on data collected in Channel 1, which is centered at 2.412 GHz. In the 5 GHz band, we concentrate on data collected in Channel 36, which is centered at 5.18 GHz. The bandwidth and center frequency of the 2.4 GHz channels are illustrated in Figure \ref{fig::802_11_channels}.

\begin{figure}[H]
	\centerline{\fbox{\includegraphics[width=0.7\textwidth]{802_11_channels.png}}}
	\caption{802.11 Wi-Fi Channels}
	\label{fig::802_11_channels}
\end{figure}

In MATLAB, we use the \texttt{sdrrx} object to collect data, and the \texttt{dsp.SpectrumAnalyzer} object to plot it. For our collects, we also turn on \texttt{PlotMaxHoldTrace} and choose a sample rate larger than the 20 MHz channel width. To ensure our collect is contiguous, we collect a single large buffer of samples. This contiguous collect ensures that we have a valid time-axis in our waterfall plots.
 
\subsection{Sweeping the Spectrum: Receiving from 60 MHz to 6 GHz}

In this experiment, we perform multiple collects of the spectrum while sweeping the center frequency in steps of the sample rate ($F_s$). Even though our instantaneous bandwidth is limited to 56 MHz, we can stitch together each of our collects to get a picture of the entire spectrum from 60 MHz to 6 GHz. This procedure is illustrated in Figure \ref{fig::sweeping_the_spectrum}. For the experiment, we specifically choose a sampling rate $F_s = 20\ \text{MHz}$. This enables us to capture the spectrum at 60 MHz even though our minimum LO frequency is 70 MHz. 

\begin{figure}[H]
	\centerline{\fbox{\includegraphics[width=0.7\textwidth]{sweeping_the_spectrum.png}}}
	\caption{Procedure for Sweeping the Spectrum}
	\label{fig::sweeping_the_spectrum}
\end{figure}

\subsection{Understanding Your Wireless Device}

In this experiment, we capture the RF spectrum of a personal RF device. We specifically choose to analyze a 2022 Honda Civic key fob. We start by finding its FCC ID. Next, using the FCC ID, we determine the key fob's operating frequency. Then, with the PlutoSDR, we capture the spectrum that results from pressing the lock button. Using the collected data, we determine the modulation scheme and propose a technique to demodulate the signal. Finally, we analyze the sequence of transmitted bits. We determine whether the sequence of bits remains consistent when holding down the lock button and whether it changes after new button presses.

\section{Results}
% Results and discussion of the laboratory experiment, including captured outputs, observations, and responses to laboratory questions.

In this section, we highlight the results for each of the experiments that we performed. We start by identifying signals with 4 different modulation techniques in the RF spectrum. Next, we examine the spectrum of 802.11 Wi-Fi signals in the 2.4 GHz and 5 GHz bands. Then, we stitch together multiple PlutoSDR collects with different LO frequencies to create a picture of the full spectrum from 60 MHz to 6 GHz. Finally, we analyze the spectrum of a 2022 Honda Civic key fob and determine a method for demodulating its transmission. We also determine whether the transmitted bits change between subsequent key presses.


\subsection{Identifying Signals in the RF Spectrum}

In this experiment, we captured the spectrum of 4 RF signals - each of which used a unique modulation technique. For our first modulation technique, we looked at the FM radio band. We specifically captured data with a center frequency of 96.1 MHz and a bandwidth of 1 MHz. Our resulting collect is shown in Figure \ref{fig::hd_radio_spectrum}.

\begin{figure}[H]
	\centerline{\fbox{\includegraphics[width=0.7\textwidth]{hd_radio_spectrum_cropped.png}}}
	\caption{HD Radio Spectrum}
	\label{fig::hd_radio_spectrum}
\end{figure}

% FM radio signal is constant amplitude and varies in frequency overtime.

Note that FM radio channel we collected includes HD radio sidebands. The HD radio sidebands are digital copies of the data transmitted using OFDM modulation. The analog radio spectrum is encoded with FM modulation and rapidly oscillates in frequency with maintaining a constant amplitude. For comparison, the OFDM digital sidebands occupy fixed bandwidth and peak at approximately constant values. This is expected because the OFDM modulation distributes power evenly across its subcarriers.

For our next collect, we looked at the spectrum for ATSC digital TV, which uses 8VSB. We specifically captured data at 485 MHz with a bandwidth of 10 MHz. This corresponds to Channel 16. Our collect is shown in Figure \ref{fig::8vsb_spectrum}.

\begin{figure}[H]
	\centerline{\fbox{\includegraphics[width=0.7\textwidth]{8vsb_spectrum_cropped.png}}}
	\caption{Channel 16 TV Spectrum}
	\label{fig::8vsb_spectrum}
\end{figure}

% 8 VSB is a form of amplitude modulation with 8 discrete values. The sinc sidelobes that result from amplitude modulation are remove with a steep filter, resulting in a moslty flat spectrum.

8VSB is a form of amplitude modulation, which uses 8 discrete amplitude levels. In 8VSB modulation, we are effectively zero-order holding 3 bit noise and modulating with a carrier. This creates a flat spectrum with sinc sidelobes centered at the carrier frequency. After filtering, we end up with the flat band-limited spectrum shown above.
 
Next, we looked at the LTE, which uses OFDM modulation. We specifically examined the 700 MHz A spectrum. To do this, we set our center frequency to 731.5 MHz and our bandwidth to 8 MHz. The resulting spectrum is shown in Figure \ref{fig::lte_spectrum}.

\begin{figure}[H]
	\centerline{\fbox{\includegraphics[width=0.7\textwidth]{lte_spectrum_cropped.png}}}
	\caption{LTE Spectrum for 700 MHz A Channel}
	\label{fig::lte_spectrum}
\end{figure}

Comparing the results to Figure \ref{fig::8vsb_spectrum}, the spectrum has much more ripple. This is expected because we are outputting tones on each of the OFDM subcarriers. Looking at time axis, we also see a lot more variation in the data due to rapidly changing symbols.

For the final modulation scheme, we examined the private carrier paging (PCP) networks, which lie between 929 and 930 MHz and use FSK modulation. Figure \ref{fig::930MHz_fsk_spectrum} shows our collected data, which is centered at 929.5 MHz with a bandwidth of 1 MHz.

\begin{figure}[H]
	\centerline{\fbox{\includegraphics[width=0.7\textwidth]{930MHz_FSK.png}}}
	\caption{PCP Spectrum Centered at 929.5 MHz}
	\label{fig::930MHz_fsk_spectrum}
\end{figure}

Examining the spectrum, we see that the page alternates between 2 discrete frequencies and maintains a constant power when it is on. This is consistent with 2-FSK modulation, which is prevalent in PCP networks.

% For the final modulation scheme, we looked at Motorola Type II communications in the 851 MHz - 869 MHz band, which uses FSK modulation according to \url{sigidwiki.com}. We referred to this modulation scheme instead of AM or GSM modulation because AM radio is out of band for the PlutoSDR and GSM is end of life. The resulting spectrum is shown in Figure \ref{fig::police_radio_spectrum}. It was captured with a center frequency of 851.4 MHz and a sample rate of 1 MHz.

%\begin{figure}[H]
%	\centerline{\fbox{\includegraphics[width=0.7\textwidth]{police_radio_spectrum_cropped.png}}}
%	\caption{Motorola Type II Spectrum}
%	\label{fig::police_radio_spectrum}
%\end{figure}

%Motorola Type II Communications leverage multiple channels that are 12.5 kHz wide. Each of these channels use a 4 FSK modulator, which maps bits to 4 discrete frequencies. Looking at the spectrum, we see that each of these channels appear almost fixed in frequency. This is because they rapidly oscillate between each of the available frequencies.

\subsection{IEEE 802.11 Wireless Local Area Network}

In this section, we display the spectrograms and waterfall plots we collected while observing the Wi-Fi bands at 2.4 GHz and 5 GHz. Figure \ref{fig::2_412_wifi_spectrum} shows a portion of the 2.4 GHz Wi-Fi spectrum, centered at 2.412 MHz with a bandwidth of 40 MHz.

\begin{figure}[H]
	\centerline{\fbox{\includegraphics[width=0.7\textwidth]{2.412g_wifi_spectrum.png}}}
	\caption{Spectrum of 802.11 Wi-Fi Signal Captured in 2.4 GHz Band}
	\label{fig::2_412_wifi_spectrum}
\end{figure}

Examining the collected data, we see that the Wi-Fi signal has a bandwidth of 20 MHz and a center frequency of 2.417 GHz. This corresponds to channel 2 in the 2.4 GHz band, which is a different channel than we were aiming to collect but sufficient for the lab. The 802.11a Wi-Fi signal we captured uses OFDM. We can see artifacts of this modulation scheme in the spectrum. For instance, we see a null DC subcarrier at the center frequency and approximately even energy across all the remaining non-zero subcarriers. We can perform similar analysis for a 5 GHz Wi-Fi signal. Figure \ref{fig::5_180g_wifi_spectrum} displays 40 MHz of the 5 GHz Wi-Fi spectrum centered at 5.18 GHz.

\begin{figure}[H]
	\centerline{\fbox{\includegraphics[width=0.7\textwidth]{5.180g_wifi_spectrum_3.png}}}
	\caption{Spectrum of 802.11 Wi-Fi Signal Captured in 5 GHz Band}
	\label{fig::5_180g_wifi_spectrum}
\end{figure}

Examining the spectrum, we see a 20 MHz wide signal centered at 5.180 GHz, which corresponds to channel 36 in the 5 GHz band. We also see energy in the adjacent channel (channel 40), which is centered at 5.20 GHz.

\subsection{Sweeping the Spectrum: Receiving from 60 MHz to 6 GHz}

In this section, we stitch together multiple collects of the PlutoSDR with LO frequencies to get a picture of the spectrum from 60 MHz to 6 GHz. As previously stated, we sweep the LO frequency in 20 MHz steps. This allows us to see the spectrum at 60 MHz even though our minimum LO frequency is 70 MHz. In between each collect, we call the \texttt{release} method of the \texttt{sdrrx} object. This prevents us from looking at stale data in the PlutoSDR receive buffers. To ensure that our transmitted signal doesn't corrupt our received data, we set the transmit LO frequency as far as possible from the receive LO frequency, max-out the attenuation, and transmit zeros. After stitching together each of our collects, we get a picture of the full PlutoSDR spectrum, which we have included in Figure \ref{fig::full_spectrum_labeled}.

\begin{figure}[H]
	\centerline{\fbox{\includegraphics[width=0.7\textwidth]{full_spectrum_labeled.png}}}
	\caption{Full Pluto SDR Spectrum with Key Bands Labeled}
	\label{fig::full_spectrum_labeled}
\end{figure}

In the spectrum, we have also labeled some of the key bands. These include FM Radio, TV, Mobile, and Wi-Fi bands. The TV bands primarily use 8VSB modulation. The Wi-Fi bands leverage OFDM in accordance with the 802.11 standard. And the mobile bands leverage mostly LTE and 5G NR.

\subsection{Understanding Your Wireless Device}

For this experiment, I examined the spectrum of my 2022 Honda Civic car key. My car key has an FCC ID of KR5TP-4. According to \url{http://fcc.io/}, this means that it transmits at 433.92 MHz. In Fosphor, I was able to tune the PlutoSDR to this operating frequency and capture the spectrum of a lock button press, which is shown in Figure \ref{fig::car_key_spectrum}.

\begin{figure}[H]
	\centerline{\fbox{\includegraphics[width=0.7\textwidth]{car_key_spectrum.png}}}
	\caption{2022 Honda Civic Key Fob Spectrum}
	\label{fig::car_key_spectrum}
\end{figure}

Using Fosphor, I also saved the collected key presses to a file, which I parsed offline in MATLAB for further analysis. In MATLAB, I was able to construct a spectrogram of the data, which is shown in Figure \ref{fig::car_key_spectrogram}.

\begin{figure}[H]
	\centerline{\fbox{\includegraphics[width=0.7\textwidth]{car_key_spectrogram.png}}}
	\caption{Spectrogram of Key Fob Data Generated in MATLAB}
	\label{fig::car_key_spectrogram}
\end{figure}

Each lock press generates 8 pulses, alternating between 2 high frequency pulses and 2 low frequency pulses. Each of these pulses use FSK modulation, which can be see in Figure \ref{fig::car_key_spectrogram_focus}

\begin{figure}[H]
	\centerline{\fbox{\includegraphics[width=0.7\textwidth]{car_key_spectrogram_focus.png}}}
	\caption{Closeup of FSK Modulation Present in Pulses}
	\label{fig::car_key_spectrogram_focus}
\end{figure}

The keyfob data we collected can be decoded using a Short-Time Fourier Transform (STFT). To decode correctly, we need to choose an FFT size which is a fraction of the bit duration. Then, if the power of the low or high frequency components is high enough above the noise floor, we know that a signal is present. Additionally, based on which FFT bin is stronger, we can declare individual bits to be 1's or 0's. For a more efficient hardware implementation, we can also use the Goertzel algorithm, which only computes the FFT output for select bins. 

The sequence of bits we receive remains constant when holding down the lock button. However, after some duration of time, the key fob stops transmitting and requires another button press to continue transmitting. For security reasons, the sequence of transmitted bits changes with new presses of the lock button. We can see this if we observe the spectrum across multiple key presses. Figure \ref{fig::car_key_spectrogram_repeat} shows the same portion of an FSK pulse as Figure \ref{fig::car_key_spectrogram_focus}. However, it comes from a different key press. Comparing the two, we can tell that the duration and ordering of the high and low frequency periods has changing. This indicates that the sequence of bits has changed.

\begin{figure}[H]
	\centerline{\fbox{\includegraphics[width=0.7\textwidth]{car_key_spectrogram_repeat.png}}}
	\caption{Repeated Collect Showing Sequence of Bits Has Changed}
	\label{fig::car_key_spectrogram_repeat}
\end{figure}
 
\section{Conclusion}

In this lab, we used the PlutoSDR to explore the RF spectrum. We started by capturing data which used 4 different modulation schemes. For this first experiment, we captured FM radio with HD radio sidebands. This demonstrated FM modulation and OFDM for the HD radio sidebands. The FM modulation had a constant amplitude and varied in frequency, while the OFDM modulation we observed was band-limited and had a nearly constant amplitude. Next, we captured ASTC digital TV, which used 8VSB modulation. This modulation scheme generated a fixed roughly flat bandwidth. Then, we captured LTE, which demonstrated OFDM modulation. In the LTE band, we had a much larger bandwidth than the HD radio sidebands which allowed us to visualize the data on individual OFDM subarriers. Finally, we observed a private carrier paging network in the 929-930 MHz band, which used FSK modulation. In the waterfall plots, this resulted in a constant amplitude signal, which hopped between two discrete frequencies.

After capturing different modulation techniques, we took a closer look at the 802.11 Wi-Fi bands, which leveraged OFDM modulation. We examined the spectrums in both the 2.4 GHz and 5 GHz bands, and we were able to capture spectral data in channel 2 (2.417 GHz) and channel 36 (5.18 GHz). In our Wi-Fi spectrums, we observed a null DC subcarrier, which is present in the 802.11 standards, and relatively constant power levels across the remaining subcarriers.

Next, we captured the full PlutoSDR spectrum from 60 MHz to 6 GHz. The PlutoSDR had a limited bandwidth of 56 MHz. However, we worked around this by sweeping the LO in steps of the sampling frequency. At each step of the sampling frequency, we captured the spectrum. Then, once we finished collecting all the data, we concatenated each of the individual spectrums to get a picture of the full PlutoSDR spectrum. In the resulting spectrum, we also labeled some of the major bands, which were used for FM radio, TV, mobile communication, and Wi-Fi.

Finally, we used the PlutoSDR to understand the spectrum of a 2022 Honda Civic key fob. We used the key fob's FCC ID to determine its center frequency of 433.92 MHz. Then, we tuned the PlutoSDR to this center frequency and collected data. Doing so, we found that the car key used FSK modulation. We proposed using the short-time Fourier transform (STFT) to decode the FSK modulation. However, for the scope of this lab, we were able to infer differences in bits by examining spectrograms. Using this approach, we found that the 2022 Honda Civic Key used rolling codes (i.e. it varied codes after each key press).

%We did this by piecing together the spectrum from multiple collects. Each of our collects captured 20 MHz of the spectrum with a different LO frequency.

%by piecing together the spectrum from multiple collects, which each used a different LO frequency. Each collect used 20 MHz of bandwidth and a different LO frequency.

% Conclusions to the overall lab that discuss meaningful lessons learned and other takeaways from the assignment. (Important)

%\nocite{analog_devices_libiio_error}
%\bibliographystyle{IEEEtran}
%\bibliography{sources}{}
%\bibliographystyle{ieeetr}
	
\end{document}