\documentclass{article}
\usepackage[margin=1in]{geometry}
\usepackage{graphicx}
\usepackage{xcolor}
\usepackage{float}
\usepackage{amsmath}
\usepackage{cite}
\usepackage{hyperref}
\usepackage{indentfirst}
\graphicspath{{..} {./images}}

\definecolor{navy-blue}{rgb}{0.22,0.38,0.71}

\renewcommand{\contentsname}{\vspace*{-2\baselineskip}}

\hypersetup{
	colorlinks,
	linkcolor=black,
	urlcolor=blue,
	citecolor=black
}
  		
\begin{document}
\begin{titlepage}
	\centering
	{\huge Lab 8 - SDR: Software Defined Radar}\\[0.25 in]
	\includegraphics[width=0.6\textwidth]{ua_logo.png}\\[0.25 in]
	{\large \textbf{ECE 531 - Software Defined Radio\\[0.25 in]
	April 28, 2025\\[0.25 in]}}
	{\large Owen Sowatzke, osowatzke@arizona.edu\\[0.05 in]
	Department of Electrical \& Computer Engineering\\[0.05 in]
	University of Arizona, Tucson, AZ 85721\\[0.5 in]}
	\hypersetup{linkcolor=navy-blue}
	\noindent\hrulefill
	\tableofcontents
	\noindent\hrulefill
\end{titlepage}

% \setlength{\parindent}{0pt}

\section{Introduction}
%Introduction to the laboratory experiment, including a brief description of the objectives and goals.

In this lab, we create a monostatic continuous wave (CW) radar using the PlutoSDR. Our CW radar uses the doppler beat frequency to measure velocity and does not measure measure range. We specifically use our CW radar to measure the angular velocity of box fan blades. Then, we explain how we can improve our radar. In the work that follows we provide the procedure for each of our experiments followed by the results. 

\section{Procedure}
% Detailed explanation of the laboratory experiment, including the design, implementation, and testing of the system.

In this section, we provide the procedures for each of our experiments. We specifically use GNU radio to implement a CW radar. A sample CW radar implementation is shown in Figure \ref{fig::gnu_radio_block_diagram}.

\begin{figure}[H]
    	\centering
    \fbox{\includegraphics[width=0.5\linewidth]{cw_radar_block_diagram.png}}
    	\caption{RF Block Diagram for a Simple CW Radar \cite{charvat_2011_build}}
    	\label{fig::cw_radar_block_diagram}
\end{figure}

\noindent CW radars, such as the one shown, use the difference between the transmitted and received frequencies to measure the doppler shift of the target.

\begin{equation*}
	f_d = f_r - f_t
\end{equation*}

\noindent For a target with a velocity $v$, the doppler shift can also be approximated as follows:

\begin{equation*}
	\label{eq::dopp_shift}
	f_d \approx \pm\frac{2v}{\lambda}
\end{equation*}

\noindent Here the positive sign corresponds to a closing (approaching) target and the negative sign corresponds to an opening (receding) target. Using the above equation and our measured doppler shifts, we can estimate the target velocity as follows:

\begin{equation*}
	v \approx \pm\frac{f_d\lambda}{2}
\end{equation*}

\noindent Using our CW radar, we specifically measure the angular velocity of a box fan. Next, we explain the strong DC return in our waterfall plots and discuss physical changes that can be made to reduce the power of this return. Then, we discuss how an interfering signal could affect the operation of our radar and propose changes to make the design more resilient to interference. Finally, we discuss how the design can be updated to support range measurements.

\section{Results}
% Results and discussion of the laboratory experiment, including captured outputs, observations, and responses to laboratory questions.

In this section, we provide the results for each of our experiments. We specifically create GNU radio flowchart which implements a CW radio in our PlutoSDR. Next, we use our design to measure the angular velocity of a box fan. Then, we discuss the short-coming of our design: DC return, resilience to interference, and no range measurement. Finally, we discuss how we can improve our design to handle these short-comings.

Our GNU radio flowchart is shown in Figure \ref{fig::gnu_radio_block_diagram}. 

\begin{figure}[H]
    	\centering
    \fbox{\includegraphics[width=0.7\linewidth]{gnu_radio_block_diagram.png}}
    	\caption{GNU Radio Flowchart which Implements a CW Radar}
    	\label{fig::gnu_radio_block_diagram}
\end{figure}

\noindent In this flowchart, we transmit a 10 kHz sinusoid on a 2.4 GHz carrier frequency. Then, we receive the sinusoid with a 500 kHz sampling rate and RF bandwidth. Note that our sampling greater is much greater than twice the highest frequency in our received signal (10 kHz plus a doppler shift), so we could further reduce our sampling rate if desired. However, if we do so, we should also reduce the RF bandwidth to reduce the amount of noise that aliases back into our spectrum. Next, we perform AGC on the received signal and signal source to maintain signal power. However, both these blocks are optional and can be removed because our transmitted signal has a fixed signal power and because we are performing AGC in the PlutoSDR. Next, we multiply the received signal by the conjugate of the transmitted signal. This is equivalent to mixing our return down to DC. After we mix down, we should be left with a tone at DC (from clutter and leakage) and a tone at the doppler shift. The doppler shift of our returns will be small with respect to our bandwidth. As such we select a sampling rate of 500 Hz for our waterfall plot, which allows us to unambiguously resolve doppler shifts up to $\pm 250\ \text{Hz}$ and radial velocities up to $\pm 15.625 \text{m}/\text{s}$. We use a low-pass filter with decimation to get down to this reduced sample rate. We also use a DC blocker to remove the clutter and leakage returns, which are much stronger than our target return.

%\begin{equation*}
%	f_d = \frac{2v}{\lambda}
%\end{equation*}

% \noindent We specifically choose  shift of 

Next, to measure the angle velocity of our box fan, we place our PlutoSDR in front of it and execute our flow chart. The results of this experiment are shown in the waterfall plot given by Figure \ref{fig::dopp_spectrum}. Note that our flowchart shows the doppler shift for three different fan speeds: slow, medium, and fast.

\begin{figure}[H]
    	\centering
    \includegraphics[width=0.9\linewidth]{dopp_spectrum2.png}
    	\caption{Waterfall Plot Displaying Doppler Shift from Box Fan}
    	\label{fig::dopp_spectrum}
\end{figure}

\noindent At these fan speeds, we measure doppler shifts of 63 Hz at low speed, 79 Hz at medimum speed, and 95 Hz at high speed. This corresponds to radial velocities of 3.9375 m/s, 4.9375 m/s, and 5.9375 m/s respectively. We have to be careful when mapping these radial velocities back to angular velocities. To help us with this analysis, we consider the setup shown in Figure \ref{fig::angular_velocity_measurement}. Note that the antenna also extends some distance $d$ out of the page.

\begin{figure}[H]
    	\centering
    \includegraphics[width=0.4\linewidth]{angular_velocity_measurement.png}
    	\caption{Measuring the Doppler Shift of a Rotating Point Target}
    	\label{fig::angular_velocity_measurement}
\end{figure}

\noindent To get the doppler shift for the point target at each point on its path of rotation, we need to compute the radial distance to the target and differentiate with respect to time. The distance, $x$ can be computed as follows:

\begin{equation*}
	x = \sqrt{r^2\cos({\omega}t)^2 + (r\sin({\omega}t) + r)^2 + d^2}
\end{equation*}

\noindent We then use MATLAB to differentiate this expression over time. The derivative of the radial distance is the radial velocity, which we can use in Equation \ref{eq::dopp_shift} to compute the corresponding doppler shift. For this analysis, we choose representative values for the distances: $r = 0.25\ \text{m}$ and $d = 0.4\ \text{m}$. We also choose $\omega = 40\pi\ \text{rad}/\text{s}$ as a placeholder for our angular velocity. The resulting doppler shifts are shown in Figure \ref{fig::dopp_shift_vs_time}.

\begin{figure}[H]
    	\centering
    \fbox{\includegraphics[width=0.5\linewidth]{dopp_shift_vs_time.png}}
    	\caption{Doppler Shift of Rotating Point Target Against Time}
    	\label{fig::dopp_shift_vs_time}
\end{figure}
 
\noindent We can make sense of this plot by examining the test setup in Figure \ref{fig::angular_velocity_measurement}. Doing so, we find that the frequency decrease as we rotate away from the antenna and increases as we rotate towards the antenna. We also create a signal with an instantaneous frequency consistent with Figure \ref{fig::dopp_shift_vs_time} and plot the FFT of it, which is shown in Figure \ref{fig::freq_response}.

\begin{figure}[H]
    	\centering
    \fbox{\includegraphics[width=0.5\linewidth]{freq_response.png}}
    	\caption{FFT of Rotating Signal Reflection}
    	\label{fig::freq_response}
\end{figure}

\noindent Note that the spectrum is very different from our hardware results shown in Figure \ref{fig::dopp_spectrum}. How can we explain the discrepancies in our results? Our simulation results were collected over one period of the sinusoid, while each point in our waterfall plot was likely collected over multiple periods of the sinusoid. If we perform our FFT over multiple cycles of the input signal, we get the plot shown in Figure \ref{fig::freq_response_multi_cycle}.

\begin{figure}[H]
    	\centering
    \fbox{\includegraphics[width=0.5\linewidth]{freq_response_multi_cycle.png}}
    	\caption{FFT of Rotating Signal Reflection Over Multiple Cycles of Rotation}
    	\label{fig::freq_response_multi_cycle}
\end{figure}

\noindent Note that repeating our signal in the time domain has inserted zeros into our frequency response similar to how upsampling in the time-domain leads to replication in the frequency domain. The frequency of the first harmonic (20 Hz) also corresponds to the frequency of rotation. We do see evidence of faint harmonics in Figure \ref{fig::dopp_spectrum}. However, we can better visualize the artifacts if we place the PlutoSDR closer to the fan, which boosts the SNR. The results of our analysis are shown in Figure \ref{fig::dopp_harmonics}.

\begin{figure}[H]
    	\centering
    \fbox{\includegraphics[width=0.75\linewidth]{dopp_harmonics.png}}
    	\caption{Data Captured with Pluto SDR Confirming Harmonics}
    	\label{fig::dopp_harmonics}
\end{figure}

\noindent In the Figure, we show the fan being turned on at the lowest speed mode and then being turned off again. As soon as the fan is turned on, we see evenly spacing harmonics confirming the results shown in Figure \ref{fig::freq_response_multi_cycle}. As a result, we should be able to the frequency of the first harmonic to directly determine the angular velocity. Note that here we must also account for the number of blades because we may be periodic over intervals smaller than a full rotation. Our fan has 5 fans. Therefore, we conclude that the angular vectors for slow, medium, and fast speeds are 63/5 rotations/s, 79/5 rotations/s, and 95/5 rotations/s respectively. This corresponds to 756 rpm, 948 rpm, and 1140 rpm, which are reasonable values for a box fan.

Two common causes of the DC return in our data are clutter (from stationary returns) and transmitter leakage. We can minimize clutter with a tighter antenna pattern or by operating in an controlled environment, such as an anechoic chamber. However, it is usually not possible for us to constrain our environment. Antenna isolation achieved through a tighter antenna pattern or other techniques can be help us achieve better antenna isolation. We can also consider other design such as pulsed doppler radars. However, these radars may not be applicable in all cases (especially with close targets). In the lab, we used a notch filter to remove DC. However, we could have also removed or ignored the offending doppler bins in post-processing.

An interfering signal would negatively impact the CW radar operation. If it was a wideband signal (such as 802.11 WiFi), it could mask the entire bandwidth of our signal and make detection more difficult. Conversely, if the interfering signal was narrow (worse case a tone), it could lead to false detections. To mitigate interference, we should try to minimize the utilized bandwidth. In the sample flowchart, provided in the lab, the RF bandwidth was larger than the sample rate. Tightening the RF bandwidth could prevent interference from aliasing into our bandwidth. Additionally, if the interference is high power, tightening our RF bandwidth can prevent the AGC from acting on interference and increase the effective number of bits for our return. We can also implement more sophisticated techniques to mitigate interference. For example, we can use a coded waveform such as a Barker code or Zadoff-chu waveform to increase the coherent gain on our returns. Additionally, having a tighter antenna pattern can reduce the interference power and increase the power of returns. If we had a ULA, we could even implement beamforming to null signals from specific directions of arrival.

We can extend our design to compute range measurements if we replace our sinusoid with a chirp. This is done in FMCW radars, for example, which we illustrate in Figure \ref{fig::fmcw_radar}

\begin{figure}[H]
    	\centering
    	\fbox{\includegraphics[width=0.6\linewidth]{FMCW Blockdiagram}}
    	\caption{Typical FMCW Block Diagram \cite{9613183}}
    	\label{fig::fmcw_radar}
\end{figure}
	
\noindent In an FMCW radar, we generate our chirp waveform with a VCO.  Typical transmit and receive signals are shown in Figure \ref{fig::fmcw_spectrogram}.

\begin{figure}[H]
    	\centering
    	\fbox{\includegraphics[width=0.6\linewidth]{FMCW Freq-Time graph}}
    	\caption{Time Frequency Plot for Transmit and Receive Signal.\cite{Long2019AssistingTV}}
    	\label{fig::fmcw_spectrogram}
\end{figure}

\noindent After mixing we get a tone with a frequency that is proportional to range. If we limit the maximum resolvable range, we can use a sample rate that is much smaller than the chirp bandwidth, which allows us to use much cheaper hardware. As shown in Figure \ref{fig::fmcw_radar}, we can then use an FFT to detect range. Additionally, we can resolve doppler by taking a slow-time FFT across pulses.

\section{Conclusion}
% Conclusions to the overall lab that discuss meaningful lessons learned and other takeaways from the assignment. (Important)

\bibliographystyle{IEEEtran}
\bibliography{sources}{}

\end{document}